\documentclass[12pt]{article}
\usepackage[hidelinks]{hyperref}
\usepackage[margin=1in]{geometry}
\usepackage{fancyhdr}
\usepackage{xcolor}
\usepackage{setspace}
\usepackage{times}
\usepackage{fontspec}
\usepackage{graphicx}
\usepackage[graphicx]{realboxes}
\usepackage[authordate, backend=biber]{biblatex-chicago}
\usepackage{amsmath}
\usepackage{float}
\addbibresource{./bdrain.bib}

\doublespacing
\pagestyle{fancy}
\fancyhf{}

\lhead{Senior Thesis Draft}
\chead{The Impact of Global Migration on Innovation}
\rhead{Jumar Martin}

\cfoot{\thepage}

\renewcommand{\thesection}{\Roman{section}} 
\renewcommand{\thesubsection}{\thesection.\Roman{subsection}}
\newcommand{\pointer}[1]{{\color{red} \center \textit{#1}}}
\hypersetup{
    pdftitle={The Impact of Global Migration on Innovation},
    pdfauthor={Jumar Martin},
    pdfsubject={Interational and Trade Economics},
    bookmarksnumbered=true,     
    colorlinks=false,            
    pdfstartview=Fit,           
    pdfpagemode=UseOutlines,
    pdfpagelayout=TwoPageRight
}
\renewcommand{\thesection}{\arabic{section}}
\renewcommand{\thesubsection}{\thesection.\arabic{subsection}}

\begin{document}
\section{Introduction}
As the world becomes connected through the democratization of international travel, the disparate cultures that dot the land collide, engage, and in time blend.
In tandem with culture, knowledge and talent also erode the political boundaries that have traditionally separated nations.
Countries find these two qualities of individuals extremely valuable and begin to actively seek and entice them to leave their home country for more opportunities, often to the detriment of their home countries; a phenomenon commonly known as the “brain drain” effect.
This effect refers to the intentional migration of highly skilled and educated people from one country to another in search of better opportunities and wages.
In the process, the source country loses its most talented individuals, which can have a negative impact on the country's economic growth and development.

Furthermore, the effect can carry potential consequences for the source country through losing their most talented citizens who could have, instead, contributed to innovation within their borders than beyond.
This is particularly relevant for countries that serve as source countries like Pakistan where the number of nationals who emigrate increase over the years.
The World Bank reported that in 2021, approximately 471,000 Pakistanis emigrated to other countries, resulting in a loss of human capital and remittances that amounts to \$31.31 billion USD per year or 9.0\% of Pakistan's 2021 GDP.
However, the existing literature suggests that source countries can partially recoup their losses through various means.

The impact of "brain drain" on source countries is still debated.
Research today touches on a bevy of reasons brain drain occurs and the impact on the innovation quotient and efficiency gains or losses for the countries on each leg of the effect.
From knowledge flows, remittances, push factors like development level and the pull factors defined by the same determinants, the intensity of the effect remains high for the highly-skilled immigrant.
In \cite{hall_brain_2005}, for instance, opts for a discussion on the theoretical aspects of the effect and the impact on the source country's GDP with a small estimation that comes with a warning to derive policy carefully.
In \citeauthor{beine_brain_2001}, the authors write that their paper is to the best of their knowledge ``the first attempt proposing an empirical validation'' of beneficial brain drain using cross-section data on 37 countries.
Most of the literature that began to explore the effect was grounded in economic theory alone.
The lack of usable and complete data sources at the time of publishing resulted in these papers not having an empirical basis or section.

With the availability of more expansive macro-level migration data from organizations such as the Organization for Economic Cooperation and Development (OECD), United Nations Educational, Scientific and Cultural Organization (UNESCO) and the International Labour Organization (ILO), researchers have been able to develop more robust empirical models to investigate the effects of brain drain on both the source and destination countries.
On the micro-level, survey data such as the Survey of Health, Ageing and Retirement in Europe and the European Community Household Panel allow for the examination of the characteristics and behaviors of individual high-skilled immigrants and their choices.
Further analysis has been conducted on the various impacts of brain drain on both destination and source countries.
Incorporating data into models has provided insight into the correlation between investing in the education of the source country and the migration rate from source to destination \autocite{beine_brain_2001}, the number of remittances in correlation with the migration of highly-skilled individuals, and the policy changes of Intellectual Property Rights (IPRs) to lure inventors to certain countries over others \autocite{mcausland_bidding_2011}.
The concept of the ability of the source country to gain even when losing human capital to destination countries—known as the ``Beneficial Brain Drain''—has also been detailed in \cite{kuhn_international_2006}.

Brain drain, while having some positive effects, can also have negative consequences for the destination country.
Economic theory suggests that an influx of highly-skilled immigrants can lead to a decrease in wages and eventually repatriation of those immigrants back to their home country.
The research of \cite{chiswick_high_2005} and \cite{hussain_reversing_2015} both show that the ``brain gain'' effect can carry a negative impact on wages in the destination country.
\cite{hussain_reversing_2015} goes further and highlights, using a life cycle model, that repatriation of immigrants can have a positive impact on the production of the home country due to the knowldege and expertise they bring back with them.

Despite the fact that the literature has been able to answer some questions, there still remain a number of questions that have yet to be answered with empirical rigor.

My research aims to deepen the understanding of the impact of brain drain on innovation, specifically by analyzing migration flow trends among countries that attract highly-skilled immigrants and the impact .
By analyzing these trends, we can measure the effect of brain drain or gain on innovation.
My findings will provide valuable insights for source and destination countries alike on how to best manage the effects of brain drain on their innovation and economic growth potential.

In order to conduct this analysis, we employ multiple linear regressions that use the Global Innovation Index from the World Intellectual Property Organization (WIPO).
The GII measures the innovation performance of countries and economies based on a variety of indicators, including research and development, education, and market sophistication.
Additionally, we will use economic data from the World Bank's World Development Indicators as independent variables.
These independent variables will include migration data, GDP per capita, education expenditure as a percentage of a country's GDP, and the ratio of employed to unemployed individuals in a country.
To study the relationship between the GII score and the independent variables, we used the following general model:

\begin{align*}
    \log(\text{Score}) = \beta_0 + \beta_1\log(\text{Edu}) + \beta_2\log(\text{EduExp}) + \beta_3\log(\text{lag.Mig}) + \beta_4\log(\text{TotEmp}) \\ + \beta_5\log(\text{TotPop}) + \epsilon
\end{align*}
This equation shows that the GII score (Score) is a function of the independent variables migration (Mig), GDP per capita (GDPperCapita), education expenditure as a percentage of GDP (EduExp\%), and the ratio of employed to unemployed individuals (Emp\%).
Migration is lagged on 1, 5, 10, and 20 year intervals.
In this regression we lag variables because in the literature brain-drain occurs over a period of time not easily shown in the short term. 

We proceed as follows.
In section \ref{literature-review}, we review the literature on brain drain and its impact on innovation.
In section \ref{data-and-methodology}, we describe the data and methodology used in this study.
In section \ref{results}, we present the results of the analysis.
In section \ref{discussion}, we discuss the results and implications of the findings.
Finally, in section \ref{conclusion}, we conclude the paper.

\section{Literature Review} \label{literature-review}
\pointer{bring in brain-drain-notes from github}
\pointer{Import literature review from overleaf}
\section{Data and Methodology} \label{data-and-methodology}

In this section, we outline the data sources and variables used in the analysis, as well as the methodology employed. We explain our rationale for selecting the data sources and variables, and provide a description on the methods used to analyze the relationship between brain drain and innovation.

\subsection{Data Sources} \label{data-sources}

We operate our models using data from the World Bank \nocite{worldbankwdi} and the World Intellectual Property Organization \nocite{gii2022}.
The World Bank hosts a database entitled the ``World Development Indicators'' which contains panel data on countries with various data points.
The data points used are defined below.

\subsection{Variables} \label{data-variables}
From the World Bank, we gathered the following data points from their World Development Indicators database for all countries:

\begin{enumerate}
    \item \textbf{SL.EMP.TOTL.SP.ZS}: Employment to population ratio, 15+, total (\% of total population ages 15+)
    \item \textbf{SE.XPD.TOTL.GD.ZS}: Education expenditure, total (\% of GDP)
    \item \textbf{NY.GDP.PCAP.CD}: GDP per capita (current US\$)
    \item \textbf{SM.POP.NETM}: Net migration
    \item \textbf{SE.TER.CUAT.BA.ZS}: Educational attainment, at least Bachelor's or equivalent, population 25+
\end{enumerate}

From the World Intellectual Property Organization, we gathered the following data points from the Global Innovation Index:
\begin{enumerate}
    \item Country Code
    \item Year
    \item Score
    \item Rank
\end{enumerate}

The indicators gathered from the WDI provide a snapshot of a country's economic and social conditions. In the context of this study, we chose these variables to represent the economic and social conditions of a country that may influence the country's innovation performance.

Utilizing the Country Code and Year, a correlation was established between the World Development Indicators and their corresponding country and year.
Due to the infeasibility of collecting patent data, the Score and Rank were employed as proxies to gauge a country's innovative aspects in comparison to others.
The score is derived from an array of indicators, encompassing research and development, education, and market diversity.
The rank represents the country's position in the GII based on its score.

\subsection{Methodology} \label{data-methodology}

Two linear regression models were developed to investigate the relationships between migration and the selected independent variables. The models contain the same Model 1 incorporates migration lagged by 1 year (lag.Mig01), while Model 2 uses migration lagged by 5 years (lag.Mig05). The following general model form is applied for both models:
\begin{align*}
\log(\text{Score}) \sim \log(\text{Edu}) + \log(\text{EduExp}) + \log(\text{TotEmp}) + \log(\text{TotPop}) + \log(\text{lag.Mig})
\end{align*}
Several statistical tests and diagnostics were performed to assess the validity and reliability of the models:
\begin{enumerate}
\item \textbf{Breusch-Pagan Test}: This test was employed to examine the presence of heteroskedasticity, a violation of the assumption of constant variance in the model's residuals.
\item \textbf{Analysis of Variance (ANOVA)}: ANOVA was used to determine the statistical significance of the independent variables in both models.
\item \textbf{Variance Inflation Factors (VIF)}: VIF values were calculated to evaluate the presence of multicollinearity among independent variables.
\end{enumerate}

\section{Results} \label{results}
\subsection{Model 1: Migration Lagged 1 Year} \label{results-model1}

The Breusch-Pagan test for homoscedasticity is employed to assess the presence of heteroskedasticity in Model 1.
The results of the test indicate that there is no evidence of heteroskedasticity, as the calculated test statistic (BP = 1.7119) has an associated p-value of 0.8874 with 5 degrees of freedom (Table \ref{tab:bptest}).
This implies that the model's residuals exhibit constant variance, a key assumption for linear regression analysis.

The ANOVA results provide valuable information about the statistical significance of the independent variables in Model 1.
As shown in Table \ref{tab:anova_mig01}, log(Edu), log(EduExp), and log(lag.Mig01) emerge as statistically significant independents of the outcome variable, with p-values substantially below the 0.05 threshold.
This suggests that these independents have a meaningful impact on the dependent variable in the model.

To assess the presence of multicollinearity among independent variables, Variance Inflation Factors (VIF) are calculated.
The VIF values for all independent variables in Model 1 are below the commonly used threshold of 10 (Table \ref{tab:vif_mig01}), indicating that multicollinearity is not a concern and that the variables are not highly correlated with each other.

Lastly, the coefficient estimates in Table \ref{tab:coef_mig01} demonstrate that an increase in log(Edu), log(EduExp), and log(lag.Mig01) is positively associated with the outcome variable.
These coefficients offer insights into the direction and magnitude of the relationship between the and the dependent variable.

\subsection{Model 2: Migration Lagged 5 Years} \label{results-model2}

For Model 2, the Breusch-Pagan test is also used to examine the presence of heteroskedasticity. The test results suggest no evidence of heteroskedasticity, as the test statistic (BP = 2.7898) has a p-value of 0.7324 with 5 degrees of freedom (Table \ref{tab:bptest}).
This confirms that the assumption of constant variance in the model's residuals is met, supporting the validity of the linear regression analysis.

The ANOVA results for Model 2 indicate that log(Edu), log(EduExp), and log(lag.Mig05) are statistically significant predictors of the outcome variable (Table \ref{tab:anova_mig05}).
This implies that these variables contribute significantly to explaining the variation in the dependent variable.

In order to evaluate the presence of multicollinearity in Model 2, VIF values are calculated for all independent variables.
The VIF values for all variables are below the threshold of 10 (Table \ref{tab:vif_Mig05}), suggesting that multicollinearity is not a concern in this model and that the indepdent variables are not highly correlated with each other.

Finally, the coefficient estimates in Table \ref{tab:coef_mig05} reveal that an increase in log(Edu), log(EduExp), and log(lag.Mig05) is positively associated with the outcome variable.
These estimates provide valuable information regarding the direction and magnitude of the relationship between the independent variables and the dependent variable in the context of the Migration Lagged 5 Years model.



\section{Discussion} \label{discussion}
\pointer{Import discussion from overleaf}
\subsection{Potential Explanations}
\pointer{Import explanations from overleaf}
\subsection{Policy Implications}
\pointer{Import implications from overleaf}
\section{Conclusion} \label{conclusion}
\pointer{Import conclusions, continued limitations, and future work  from overleaf}
\section{Appendix} \label{appendix}
\begin{table}[H]
    \centering
    \begin{tabular}{lrrr}
        \hline
                  & BP     & df & p-value \\
        \hline
        lag.Mig01 & 1.7119 & 5  & 0.8874  \\
        lag.Mig05 & 2.7898 & 5  & 0.7324  \\
        \hline
    \end{tabular}
    \caption{Breusch-Pagan Test for Homoscedasticity}
    \label{tab:bptest}
\end{table}

\begin{table}[H]
    \centering
    \begin{tabular}{lrrr}
        \hline
                       & Df  & F           & Pr($>$F)                 \\
        \hline
        log(Edu)       & 1   & 83.46001695 & 0.0000000000000001766604 \\
        log(EduExp)    & 1   & 18.47129065 & 0.0000288108433659378874 \\
        log(lag.Mig01) & 1   & 17.48836815 & 0.0000459725429858645770 \\
        log(TotEmp)    & 1   & 0.04654342  & 0.8294466282533620171691 \\
        log(TotPop)    & 1   & 4.03937910  & 0.0460132398722408603176 \\
        Residuals      & 172 & NA          & NA                       \\
        \hline
    \end{tabular}
    \caption{ANOVA Results for the Migration Lagged 1 Year Model}
    \label{tab:anova_mig01}
\end{table}


\begin{table}[H]
    \centering
    \begin{tabular}{lrrr}
        \hline
                       & Df  & F           & Pr($>$F)                      \\
        \hline
        log(Edu)       & 1   & 120.7882431 & 0.000000000000000000001848827 \\
        log(EduExp)    & 1   & 25.8978526  & 0.000000966154129799688360004 \\
        log(lag.Mig05) & 1   & 20.2608308  & 0.000012669510039792481576829 \\
        log(TotEmp)    & 1   & 0.4813491   & 0.488781594513344797015008680 \\
        log(TotPop)    & 1   & 3.0258827   & 0.083799295093312575755106764 \\
        Residuals      & 166 & NA          & NA                            \\
        \hline
    \end{tabular}
    \caption{ANOVA Results for the Migration Lagged 5 Years Model}
    \label{tab:anova_mig05}
\end{table}

\begin{table}[H]
    \centering
    \begin{tabular}{lrrrrr}
        \hline
            & log(Edu) & log(EduExp) & log(lag.Mig01) & log(TotEmp) & log(TotPop) \\
        \hline
        VIF & 1.169    & 1.204       & 1.809          & 1.199       & 1.773       \\
        \hline
    \end{tabular}
    \caption{Variance Inflation Factors for the Migration Lagged 1 Year Model}
    \label{tab:vif_mig01}
\end{table}


\begin{table}[H]
    \centering
    \begin{tabular}{lrrrrr}
        \hline
            & log(Edu) & log(EduExp) & log(lag.Mig05) & log(TotEmp) & log(TotPop) \\
        \hline
        VIF & 1.113    & 1.156       & 1.903          & 1.356       & 1.823       \\
        \hline
    \end{tabular}
    \caption{Variance Inflation Factors for the Migration Lagged 5 Years Model}
    \label{tab:vif_Mig05}
\end{table}

\begin{table}[H]
    \centering
    \begin{tabular}{lrrrr}
        \hline
                       & Estimate  & Std. Error & z value & Pr($>|z|$)                  \\
        \hline
        (Intercept)    & 2.479082  & 0.518935   & 4.7773  & 0.000001777 ***             \\
        log(Edu)       & 0.290411  & 0.031789   & 9.1356  & $<$ 0.00000000000000022 *** \\
        log(EduExp)    & 0.201943  & 0.046987   & 4.2978  & 0.000017248 ***             \\
        log(lag.Mig01) & 0.043928  & 0.010504   & 4.1819  & 0.000028907 ***             \\
        log(TotEmp)    & 0.022661  & 0.105037   & 0.2157  & 0.82919                     \\
        log(TotPop)    & -0.024308 & 0.012094   & -2.0098 & 0.04445 *                   \\
        \hline
    \end{tabular}
    \caption{Coefficient Estimates for the Migration Lagged 1 Year Model}
    \label{tab:coef_mig01}
\end{table}

\begin{table}[H]
    \centering
    \begin{tabular}{lrrrr}
        \hline
                       & Estimate   & Std. Error & z value & Pr($>|z|$)                  \\
        \hline
        (Intercept)    & 2.9228886  & 0.5190316  & 5.6314  & 0.00000001787 ***           \\
        log(Edu)       & 0.2531685  & 0.0230355  & 10.9904 & $<$ 0.00000000000000022 *** \\
        log(EduExp)    & 0.2134526  & 0.0419440  & 5.0890  & 0.00000035997 ***           \\
        log(lag.Mig05) & 0.0421024  & 0.0093536  & 4.5012  & 0.00000675698 ***           \\
        log(TotEmp)    & -0.0715674 & 0.1031538  & -0.6938 & 0.48781                     \\
        log(TotPop)    & -0.0201290 & 0.0115717  & -1.7395 & 0.08195 .                   \\
        \hline
    \end{tabular}
    \caption{Coefficient Estimates for the Migration Lagged 5 Years Model}
    \label{tab:coef_mig05}
\end{table}

% \begin{table}[t]
%     \centering
%     \begin{tabular}{lrrrr}
%         \hline
%                      & Estimate    & Std. Error  & t value  & Pr($>|t|$)                 \\
%         \hline
%         (Intercept)  & 16.8460     & 3.9495      & 4.265    & 0.0000276 ***              \\
%         lag.Mig01    & 0.000010150 & 0.000002526 & 4.018    & 0.0000762 ***              \\
%         lag.Edu01    & 0.7502      & 0.0565      & 13.284   & $<$ 0.0000000000000002 *** \\
%         lag.TotEmp01 & $-$0.0236   & 0.0578      & $-$0.408 & 0.683                      \\
%         lag.EduExp01 & 2.2405      & 0.3311      & 6.766    & 0.0000000000813 ***        \\
%         \hline
%     \end{tabular}
%     \caption{Regression with lagged variables}
%     \label{tab:lagged_vars}
% \end{table}

% \begin{table}[t]
%     \centering
%     \begin{tabular}{lrrrr}
%         \hline
%                     & Estimate        & Std. Error   & t value  & Pr($>|t|$)                 \\
%         \hline
%         (Intercept) & 25.4192         & 1.0844       & 23.441   & $<$ 0.0000000000000002 *** \\
%         Mig         & 0.0000196087    & 0.0000076642 & 2.558    & 0.011 *                    \\
%         Edu         & 0.7644          & 0.0517       & 14.793   & $<$ 0.0000000000000002 *** \\
%         Mig:Edu     & $-$0.0000002212 & 0.0000002634 & $-$0.840 & 0.402                      \\
%         \hline
%     \end{tabular}
%     \caption{Regression with interaction term between migration and education}
%     \label{tab:mig_edu_interaction}
% \end{table}

% \begin{table}[ht]
%     \centering
%     \begin{tabular}{lrrrr}
%         \hline
%                     & Estimate       & Std. Error  & t value  & Pr($>|t|$)                 \\
%         \hline
%         (Intercept) & 28.3871        & 1.1086      & 25.606   & $<$ 0.0000000000000002 *** \\
%         Mig         & $-$0.000007655 & 0.000004502 & $-$1.700 & 0.0894 .                   \\
%         EduExp      & 1.7438         & 0.2279      & 7.651    & $<$ 0.0000000000000493 *** \\
%         Mig:EduExp  & 0.000005487    & 0.000001115 & 4.922    & $<$ 0.0000010103192624 *** \\
%         \hline
%     \end{tabular}
%     \caption{Regression with interaction term between migration and education expenditure}
%     \label{tab:model5}
% \end{table}



\printbibliography
\end{document}