\documentclass[12pt]{article}
\usepackage[hidelinks]{hyperref}
\usepackage[margin=1in]{geometry}
\usepackage{fancyhdr}
\usepackage{xcolor}
\usepackage{setspace}
\usepackage{times}
\usepackage{fontspec}
\usepackage{graphicx}
\usepackage[graphicx]{realboxes}
\usepackage[authordate, backend=biber]{biblatex-chicago}
\addbibresource{./bdrain.bib}

\doublespacing
\pagestyle{fancy}
\fancyhf{}

\lhead{ECO 4800}
\chead{Thesis Proposal: In the Face of Globalization}
\rhead{Jumar Martin}

\cfoot{\thepage}

\renewcommand{\thesection}{\Roman{section}} 
\renewcommand{\thesubsection}{\thesection.\Roman{subsection}}
\newcommand{\pointer}[1]{{\color{red} \center \textit{#1}}}
\hypersetup{
    pdftitle={Dissertation Proposal: In the Face of Globalization},
    pdfauthor={Jumar Martin},
    pdfsubject={Interational and Trade Economics},
    bookmarksnumbered=true,     
    colorlinks=false,            
    pdfstartview=Fit,           
    pdfpagemode=UseOutlines,
    pdfpagelayout=TwoPageRight
}

\begin{document}
\section{Introduction}
\pointer{Defining the nature of the problem}
As the world becomes connected through the democratization of international travel, the disparate cultures that dot the land collide, engage, and in time blend.
The flow of knowledge and talent across borders undermines traditional political boundaries between nations.
Countries actively recruit highly skilled and educated individuals, often causing a loss of talent in their home countries, known as the "brain drain" effect.
The "brain drain" effect refers to the migration of highly skilled and educated individuals to other countries for better opportunities and wages.
In the process, the source country loses its most talented individuals, which can have a negative impact on the country's economic growth and development.

The "brain drain" effect can have negative consequences for the source country, as it loses its most talented citizens who could have contributed to innovation within the country.
This is particularly relevant for countries that serve as source countries like Pakistan where the number of nationals who emigrate increase over the years.
The World Bank reported that in 2021, 471,000 Pakistanis emigrated, resulting in a loss of human capital and \$31.31 billion USD in remittances, equivalent to 9.0\% of Pakistan's GDP.
However, the existing literature suggests that source countries can partially recoup their losses through various means.

\pointer{Paragraphs 2-4: Casting the problem as an economic question (at least two, maybe 3 paragraphs depending on the topic)}

The impact of "brain drain" on source countries is still debated.
Research today touches on a bevy of reasons brain drain occurs and the impact on the innovation quotient and efficiency gains or losses for the countries on each leg of the effect.
From knowledge flows, remittances, push factors like development level and the pull factors defined by the same determinants, the intensity of the effect remains high for the highly-skilled immigrant.
In \citeauthor{hall_brain_2005}, for instance, opts for a discussion on the theoretical aspects of the effect and the impact on the source country's GDP with a small estimation that comes with a warning to derive policy carefully. In \citeauthor{beine_brain_2001}, the authors write that their paper is to the best of their knowledge ``the first attempt proposing an empirical validation'' of beneficial brain drain using cross-section data on 37 countries.
Early studies on the "brain drain" effect focused mainly on economic theory.
The lack of usable and complete data sources at the time of publishing resulted in these papers not having an empirical basis or section.

With the availability of more expansive macro-level migration data from organizations such as the Organization for Economic Cooperation and Development (OECD), United Nations Educational, Scientific and Cultural Organization (UNESCO) and the International Labour Organization (ILO), researchers have been able to develop more robust empirical models to investigate the effects of brain drain on both the source and destination countries.
On the micro-level, survey data such as the Survey of Health, Ageing and Retirement in Europe and the European Community Household Panel allow for the examination of the characteristics and behaviors of individual high-skilled immigrants and their choices.
Further analysis has been conducted on the various impacts of brain drain on both destination and source countries.
Incorporating data into models has provided insight into the correlation between investing in education and migration rates, the relationship between remittances and migration of highly-skilled individuals, and the impact of Intellectual Property Rights policies on the migration of inventors. \autocite{beine_brain_2001}, \autocite{mcausland_bidding_2011}
The ability of the source country to gain even when losing human capital to destination countries—known as the ``Beneficial Brain Drain''—has also been detailed in \citeauthor{kuhn_international_2006}.

Brain drain, while having some positive effects, can also have negative consequences for the destination country.
Economic theory suggests that an influx of highly-skilled immigrants can lead to a decrease in wages and eventually repatriation of those immigrants back to their home country.
The research of \citeauthor{chiswick_high_2005} and \citeauthor{hussain_reversing_2015} both show that the ``brain gain'' effect can carry a negative impact on wages in the destination country.
\citeauthor{hussain_reversing_2015} goes further and highlights, using a life cycle model, that repatriation of immigrants can have a positive impact on the production of the home country due to the knowldege and expertise they bring back with them. 
Despite the fact that the literature has been able to answer some questions, there still remain a number of questions that have yet to be answered with empirical rigor.

% \pointer{rewrite}
% {\color{gray}
% With the availability of more expansive macro-level migration data from organizations such as the Organization for Economic Cooperation and Development (OECD), United Nations Educational, Scientific and Cultural Organization (UNESCO) and the International Labour Organization (ILO), researchers have been able to develop more robust empirical models to investigate the effects of brain drain on both the source and destination countries.
% These models utilize data from a variety of sources, including micro-level survey data such as the Survey of Health, Ageing and Retirement in Europe and the European Community Household Panel to examine the characteristics and behaviors of individual high-skilled immigrants and their decision-making processes.

% The incorporation of this data into models has enabled researchers to delve deeper into the intricacies of the relationship between brain drain and economic growth, and has provided new insights into the significance of factors such as investment in education in the source country, the correlation between remittances and the migration of highly-skilled individuals, and the impact of Intellectual Property Rights (IPRs) policies on the migration of inventors.
% }
\pointer{Define your research question (How impactful is brain drain on innovation?)}

% With brain drain being found as a significant effect in the literature, we're interested in developing a further understanding of its impact on innovation in both source and destination countries.
% Since immigrants, in general, can find jobs in their destination countries that provide enough to remit a portion of their wages, impacting the GDP of their home country, isolating for the impact highly-skilled immigrants is interesting as we can measure the improvement in innovation or lack thereof that occur when a country is affected by the effect in contrast with the destination country of their departing inventors.
% Countries that are usually subject to the brain drain effect should not be relegated to being lower developed forever; evaluating whether the effect furthers this experience or, instead, retards such is important for suggesting policy in these countries.
% On the other side of the effect, the paper will also inform destination countries on whether to continue appealing to high-skilled workers as well and which policies work best in winning over these individuals.

We aim to gain a deeper understanding of brain drain's impact on innovation in both source and destination countries. By focusing on the impact of highly-skilled immigrants specifically, we can measure the improvement or lack thereof in innovation when a country experiences brain drain in contrast to the destination country. This information can help countries that are often subject to brain drain to develop policies that will prevent them from being stuck in underdevelopment. Additionally, the research can inform destination countries on whether they should continue to attract high-skilled workers and what policies are effective in doing so.


% With brain drain being recognized as a significant phenomenon in the literature, we are interested in gaining a deeper understanding of its impact on innovation in both source and destination countries.
% By isolating the impact of highly-skilled immigrants specifically, we can measure the improvement or lack thereof in innovation that occurs when a source country is affected by brain drain in contrast to the destination country where the immigrants are relocating.

% It is important to note that countries that are usually subject to brain drain should not be resigned to being permanently lower developed.
% Evaluating whether the effect furthers or retards this experience is crucial for suggesting effective policies for these countries. On the other side of the equation, the research will also inform destination countries on whether to continue attracting highly-skilled workers and which policies work best in attracting them.
% The study will provide insights on how to balance the positive and negative effects of brain drain, to ensure that both source and destination countries can benefit from it.

% The research aims to gain a deeper understanding of the impact of brain drain on innovation in both source and destination countries.
% By isolating the impact of highly-skilled immigrants specifically, the study aims to measure the improvement or lack thereof in innovation that occurs when a source country is affected by brain drain in contrast to the destination country where the immigrants are relocating.
% The research also aims to evaluate how brain drain affects the development of the source country, and how it can inform policies for both source and destination countries to attract and retain highly-skilled immigrants in a way that balances the positive and negative effects of brain drain.
% In essence, this research seeks to answer the question of how significant the impact of brain drain is on innovation.

\pointer{Identify probable data source and sample}

\printbibliography
\end{document}