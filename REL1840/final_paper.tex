\documentclass[12pt]{article}
\usepackage{hyperref,times,setspace,biblatex}

\addbibresource{sources.bib}

\usepackage[margin=1in]{geometry}
\author{Jumar Martin}
\title{The Early Ecumenical Councils and the Reformers: A Comparative Analysis}
\date{\today}

\doublespacing

\begin{document}

\maketitle
\tableofcontents
\newpage

\section{Introduction}
 In the history of Christianity, numerous movements and prominent figures have emerged with the purpose of elucidating and delineating the faith's fundamental tenets.
 Among these, the Early Ecumenical Councils and the Reformers stand as two noteworthy instances.
 The former refers to assemblies of bishops and theologians convened to resolve doctrinal disputes within the church, while the latter encompasses a group of individuals who aimed to reform the Catholic Church during the sixteenth century.
 Despite distinctions in their historical contexts and methodologies, both movements shared a common objective: to refine Christian doctrine and mold the convictions and practices of adherents.
 This paper will investigate the ways in which the Early Ecumenical Councils and the Reformers contributed to the explication of Christian doctrine and the evolution of the church.

\section{The Early Ecumenical Councils}
\textit{The Early Ecumenical Councils were assemblies of bishops and theologians convened to resolve doctrinal disputes within the church.
The Early Ecumenical Councils, such as the Councils of Nicaea, Constantinople, and Ephesus, dealt with issues such as the nature of Christ, the Trinity, and the relationship between the Father and the Son}
\section{The Reformers}
\section{Comparative Analysis}
\printbibliography

\end{document}
